% Conclusion
\section{Conclusion}
\subsection{Full workflow}
\begin{frame}[fragile]{Full workflow}
    \makebox[\textwidth][c]{
    \begin{tikzpicture}[
            node distance=0.8cm,
            >={Triangle[angle=60:1pt 2]},
            shorten >= 2pt,
            shorten <= 2pt,
            arrow/.style={
                ->,
                mLightBrown,
                line width=3pt
            }
        ]
        \ImageNode[label={Create Project}]{A}{python_logo.png}
        \ImageNode[label={Configure},right=of A]{B}{python_logo.png}
        \ImageNode[label={0:Code},right=of B]{C}{python_logo.png}
        \ImageNode[label={Upgrade Version},above right=of C]{D}{python_logo.png}
        \ImageNode[label={Package},right=of D]{E}{python_logo.png}
        \ImageNode[label={PyPI},right=of E]{F}{python_logo.png}

        \ImageNode[label={-90:GitHub},below right=of C]{G}{github_logo.png}
        \ImageNode[label={-90:Travis},right=of G]{H}{travis_logo.png}
        \ImageNode[label={-90:Codecov},right=of H]{I}{codecov_logo.png}
        \ImageNode[label={0:ReadTheDocs},below=of H]{J}{readthedocs_logo.png}

        \draw[arrow] (A) -- (B);
        \draw[arrow] (B) -- (C);
        \draw[arrow] (C) -- (D);
        \draw[arrow] (D) -- (E);
        \draw[arrow] (E) -- (F);
        \draw[arrow] (C) -- (G);
        \draw[arrow] (G) -- (H);
        \draw[arrow] (H) -- (I);
        \draw[arrow] (G) -- (J);

        \onslide<2->{
            \node (X) [draw=mLightGreen, fit= (A) (B), inner sep=0.55cm, thick, fill=mLightGreen, fill opacity=0.1] {};
            \node [yshift=1.5ex, mLightGreen] at (X.south) {Creation};
        }
        \onslide<3->{
            \node (Y) [draw=mLightGreen, fit= (D) (E) (F), inner sep=0.55cm, thick, fill=mLightGreen, fill opacity=0.1] {};
            \node [yshift=1.5ex, mLightGreen] at (Y.south) {Packaging \& Distributing};
        }
        \onslide<4->{
            \node (Z) [draw=mLightGreen, fit= (G) (H) (I) (J), inner sep=0.55cm, thick, fill=mLightGreen, fill opacity=0.1] {};
            \node [yshift=1.5ex, mLightGreen] at (Z.south) {Continuous Integration};
        }
    \end{tikzpicture}
    }
    \note {
        El desarrollo de una nueva funcionalidad se podría dividir en tres módulos:
        \begin{enumerate}
            \item Creación.
            \item Empaquetado y distribución.
            \item Integración continua.
        \end{enumerate}
    }
\end{frame}

\subsection{Simplified workflow}
\begin{frame}[fragile]{Simplified workflow I}
    \makebox[\textwidth][c]{
    \begin{tikzpicture}[
            node distance=0.8cm,
            >={Triangle[angle=60:1pt 2]},
            shorten >= 2pt,
            shorten <= 2pt,
            arrow/.style={
                ->,
                mLightBrown,
                line width=3pt
            }
        ]
        \ImageNode[label={Cookiecutter}]{A}{python_logo.png}
        \ImageNode[label={0:Code},right=of A]{B}{python_logo.png}
        \ImageNode[label={Clinner Build},above right=of B]{C}{python_logo.png}

        \ImageNode[label={-90:GitHub},below right=of B]{G}{github_logo.png}
        \ImageNode[label={-90:Travis},right=of G]{H}{travis_logo.png}
        \ImageNode[label={-90:Codecov},right=of H]{I}{codecov_logo.png}
        \ImageNode[label={0:ReadTheDocs},below=of H]{J}{readthedocs_logo.png}

        \draw[arrow] (A) -- (B);
        \draw[arrow] (B) -- (C);
        \draw[arrow] (B) -- (G);
        \draw[arrow] (G) -- (H);
        \draw[arrow] (H) -- (I);
        \draw[arrow] (G) -- (J);
    \end{tikzpicture}
    }
    \note{
        Los módulos anteriores son sustituidos por:
        \begin{enumerate}
            \item Creación, reemplazado por cookiecutter.
            \item Empaquetado y distribución, reemplazado por clinner.
            \item Integración continua, automático.
        \end{enumerate}
    }
\end{frame}

\begin{frame}[fragile]{Simplified workflow II}
    \begin{block}{Workflow execution}
        \begin{minted}{bash}
cookiecutter <project_name>
...code...
python build.py dist (patch|minor|major)
git push
        \end{minted}
        \note{
            \begin{enumerate}
                \item Crea una vez el proyecto con cookicutter.
                \item Desarrolla una nueva funcionalidad.
                \item Sube de versión, empaqueta y distribuye con un comando.
                \item Pushea tu código.
            \end{enumerate}

            Se dedica mucho más tiempo a desarrollar ya que todo el proceso de empaquetado y distribución es casi automático.
        }
    \end{block}
\end{frame}


