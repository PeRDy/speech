% Building a Machine Learning Service
\section{Building a Machine Learning Service}
\begin{frame}{How to expose the ML model?}
    Machine Learning models can be used either as an internal piece of a service or as a service itself.

    \pause

    If it is used as an \textbf{internal piece} you won't notice it, such as scoring or recommendation systems within bigger products like Spotify or Netflix.

    \pause

    But you can also find them as a \textbf{service that exposes an API} to directly interact with the model. There are many examples of that in AWS, Google Cloud, Azure...
\end{frame}

\begin{frame}{Wrapping up a ML model}
    One of the most widely adopted way of serving a ML model is to wrap it into a \textbf{REST API} with specific methods for calling the model.
\end{frame}

\begin{frame}[fragile]{The Service}
    Our service will expose a single endpoint that let us interact with the model.

    \begin{exampleblock}{Request}
        {
            \footnotesize
            \begin{description}
                \item[Verb] \texttt{\textbf{GET}}
                \item[URL] \texttt{https://service.url/analyze/}
                \item[Params] \texttt{text=The\%20girl\%20is\%20having\%20fun\%20while\%20playing}
            \end{description}
        }
    \end{exampleblock}

    \begin{exampleblock}{Response}
        \begin{minted}[fontsize=\footnotesize]{js}
{
  "text": "The girl is having fun while playing",
  "sentiment": "Positive",
  "score": 0.6321590542793274
}
        \end{minted}
    \end{exampleblock}
\end{frame}

\begin{frame}{Building a REST API with Flama}
    To build a REST API we need to define:

    \begin{enumerate}[<+->]
        \item A \textbf{component} that loads our ML model.
        \item The \textbf{data schema} for our response.
        \item The \textbf{view} function that will be called through requests to \texttt{/analyze/} endpoint.
        \item The whole API \textbf{application}.
    \end{enumerate}

    \only<4>{
        Everything put together is less than 100 lines of python code.
    }
\end{frame}

\begin{frame}[fragile]{ML Component}
    \begin{minted}[fontsize=\footnotesize]{python}
class SentimentAnalysisModel:
    def __init__(self, model, words: typing.Dict[str, int]):
        self.model = model
        self.words = words

    def predict(self, text: str) -> typing.Tuple[float, str]:
        x = text.lower().split()
        x = [self.words.get(i, 0)
             if self.words.get(i, 0) <= VOCABULARY_LENGTH else 0 for i in x]
        x = sequence.pad_sequences([x], maxlen=MAX_WORDS)
        score = self.model.predict(x)
        sentiment = "Positive" if self.model.predict_classes(x)[0][0] == 1 \
            else "Negative"

        return score, sentiment

class SentimentAnalysisModelComponent(Component):
    def __init__(self, model_path: str, words_path: str):
        self.model = load_model(model_path)
        self.model._make_predict_function()
        with open(words_path) as f:
            self.words = json.load(f)

    def resolve(self) -> SentimentAnalysisModel:
        return SentimentAnalysisModel(model=self.model, words=self.words)
    \end{minted}
\end{frame}

\begin{frame}[fragile]{Data Schema}
    \begin{minted}[fontsize=\footnotesize]{python}
class SentimentAnalysis(Schema):
    text = fields.String(
        title="text",
        description="Text to analyze"
    )
    score = fields.Float(
        title="score",
        description="Sentiment score in range [0,1]"
    )
    sentiment = fields.String(
        title="sentiment",
        description="Sentiment class (Positive or Negative)"
    )
    \end{minted}
\end{frame}

\begin{frame}[fragile]{Analysis View}
    \begin{minted}[fontsize=\footnotesize]{python}
def analyze(text: str, model: SentimentAnalysisModel) -> SentimentAnalysis:
    """
    tags:
        - sentiment-analysis
    summary:
        Sentiment analysis.
    description:
        Performs a sentiment analysis on a given text.
    responses:
        200:
            description: Analysis result.
    """
    text = unquote(text)
    score, sentiment = model.predict(text)

    return {
        "text": text,
        "score": score,
        "sentiment": sentiment,
    }
    \end{minted}
\end{frame}

\begin{frame}[fragile]{API Application}
    \begin{minted}[fontsize=\footnotesize]{python}
app = Flama(
    components=[SentimentAnalysisModelComponent("model.h5", "words.json")],
    title="Sentiment Analysis",
    version="0.1",
    description="A sentiment analysis API for movies reviews",
)

app.add_route("/analyze/", analyze, methods=["GET"])
    \end{minted}
\end{frame}
