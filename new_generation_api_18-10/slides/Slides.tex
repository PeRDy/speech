\documentclass[final, 9pt, svgnames]{beamerPerdy}
\usetheme[
    titleformat title=smallcaps,
    progressbar=frametitle,
    numbering=fraction
]{metropolis}

\definecolor{Teal}{HTML}{008080}

\usetikzlibrary{positioning,fit,arrows.meta}

%\setbeameroption{show notes on second screen=right}

\setdefaultlanguage{english}

% Comando para usar comentarios
\newcommand{\comment}[1]{\textcolor{comment}{\footnotesize{#1}\normalsize}}

\newcommand\ImageNode[3][]{
    \node[draw=mLightBrown,line width=1pt,#1] (#2) {\includegraphics[width=1cm,height=1cm]{#3}};
}


\title{New Generation of APIs}
\author[J. A. Perdiguero López]{
José Antonio Perdiguero López\\
\href{http://www.perdy.io}{\scriptsize{\faGlobe\; http://www.perdy.io}}\\
\href{https://github.com/PeRDy}{\scriptsize{\faGithub\; https://github.com/PeRDy}}\\
\href{https://www.linkedin.com/in/perdy}{\scriptsize{\faLinkedin\; https://www.linkedin.com/in/perdy}}\\
\href{mailto://perdy@perdy.io}{\scriptsize{\faAt\; perdy@perdy.io}}}
\institute[Whalar]{Head of Data Science @ Whalar}
\date{October 6, 2018}

\begin{document}

% Title
\begin{frame}[noframenumbering, plain]
    \titlepage
\end{frame}

% Index
\begin{frame}[noframenumbering, plain]{Index}
    \setbeamertemplate{section in toc}[circle]
    \tableofcontents
\end{frame}

% Introduction 
\section{Introduction}
\begin{frame}{Introduction}
    What's wrong with current API frameworks? Nothing at all, except they seems a bit \textbf{old} and \textbf{unexpressive}.

    Let's improve it using new Python functionalities like:
    
    \begin{itemize}
        \item Type annotation.
        \item Module \texttt{typing}. 
        \item New \texttt{async}/\texttt{await} semantic.
        \item Module \texttt{asyncio}.
    \end{itemize}
\end{frame}

\begin{frame}{Introduction}
    The codebase can be very \textbf{expressive} in terms of describing the API. Make it the first information source of the API.
\end{frame}

\begin{frame}{Goals}
    \textbf{Speep up} API building and maintenance.

    Create an API with a codebase \textbf{expressive}.

    An \textbf{interactive documentation} kept \textbf{in sync} with the API.

    A way to infer and generate API \textbf{schema}.

    Possibility to work with \textbf{ASGI} and \textbf{websockets}.
\end{frame}

\begin{frame}{Example API}
    Puppy API \faPaw

    \begin{itemize}
        \item \textbf{Register} a new puppy.
        \item \textbf{List} all puppies, filtered by name.
    \end{itemize}
\end{frame}

\begin{frame}[fragile]{Example API}
    \begin{minted}[fontsize=\footnotesize]{python}
def puppy(request):
    if request.method == "POST":
        data = JSONParser().parse(request)
        serializer = PuppySerializer(data=data)
        if serializer.is_valid():
            serializer.save()
            return JsonResponse(serializer.data, status=201)
        return JsonResponse(serializer.errors, status=400)

    if request.method == "GET":
        puppies = Puppy.objects.all()
        name = request.query_params.get('name', None)
        if name is not None:
            puppies = puppies.filter(name=name)
        serializer = PuppySerializer(puppies, many=True)
        return JsonResponse(serializer.data, safe=False)
    \end{minted}
\end{frame}

% Routes 
\section{Routes}
\begin{frame}[fragile]{A single route for a single view}
    \begin{minted}[fontsize=\footnotesize]{python}
urls = [
    ("/puppy/", puppy),
]
    \end{minted}
\end{frame}

\begin{frame}[fragile]{Splitting the view}
    \begin{minted}[fontsize=\footnotesize]{python}
def register_puppy(request):
    """
    Register a new puppy !
    """
    data = JSONParser().parse(request)
    serializer = PuppySerializer(data=data)
    if serializer.is_valid():
        serializer.save()
        return JsonResponse(serializer.data, status=201)
    return JsonResponse(serializer.errors, status=400)

def list_puppy(request):
    """
    List all puppies !
    """
    puppies = Puppy.objects.all()
    name = request.query_params.get('name', None)
    if name is not None:
        puppies = puppies.filter(name=name)
    serializer = PuppySerializer(puppies, many=True)
    return JsonResponse(serializer.data, safe=False)
    \end{minted}
\end{frame}

\begin{frame}[fragile]{Multiple routes for multiple views}
    \begin{minted}[fontsize=\footnotesize]{python}
urls = [
    ("/puppy/", "POST", register_puppy),
    ("/puppy/", "GET", list_puppy),
]
    \end{minted}
\end{frame}

% Views
\section{Views}
\begin{frame}[fragile]{Async view}
    \begin{minted}[fontsize=\footnotesize]{python}
async def register_puppy(request):
    """
    Register a new puppy !
    """
    data = JSONParser().parse(request)
    serializer = PuppySerializer(data=data)
    if serializer.is_valid():
        serializer.save()
        # Do your async stuff...
        return JsonResponse(serializer.data, status=201)
    return JsonResponse(serializer.errors, status=400)

def list_puppy(request):
    """
    List all puppies !
    """
    puppies = Puppy.objects.all()
    name = request.query_params.get('name', None)
    if name is not None:
        puppies = puppies.filter(name=name)
    serializer = PuppySerializer(puppies, many=True)
    return JsonResponse(serializer.data, safe=False)
    \end{minted}
\end{frame}



% Components
\section{Components}
\begin{frame}[fragile]{Validation as part of the view}
    \begin{minted}[fontsize=\footnotesize]{python}
def list_puppy(request: Request) -> Response:
    """
    List all puppies !
    """
    puppies = Puppy.objects.all()
    name = request.query_params.get('name', None)
    if name is not None:
        if name[0].islower():
            raise ValidationError("Puppy name must start with uppercase")
        else:
            puppies = puppies.filter(name=name)
    serializer = PuppySerializer(puppies, many=True)
    return JsonResponse(serializer.data, safe=False)
    \end{minted}
\end{frame}

\begin{frame}[fragile]{Validation as part of component instance}
    \begin{minted}[fontsize=\footnotesize]{python}
class PuppyName:
    def __init__(self, name: QueryParam):
        if name[0].islower():
            raise ValidationError("Puppy name must start with uppercase")

        self.value = name
    \end{minted}
\end{frame}

\begin{frame}[fragile]{View with injected components}
    \begin{minted}[fontsize=\footnotesize]{python}
def list_puppy(name: PuppyName) -> Response:
    """
    List all puppies !
    """
    puppies = Puppy.objects.all()
    name = request.query_params.get('name', None)
    if name is not None:
        puppies = puppies.filter(name=name.value)
    serializer = PuppySerializer(puppies, many=True)
    return JsonResponse(serializer.data, safe=False)
    \end{minted}
\end{frame}

% Types
\section{Types}
\begin{frame}[fragile]{Types to define Schemas}
    \begin{minted}[fontsize=\footnotesize]{python}
class PuppyType(Type):
    name = validators.String(
        title="name",
        description="Word to pay attention"
    )
    age = validators.Integer(
        title="age",
        description="I'm a puppy yet?"
    )
    \end{minted}
\end{frame}

\begin{frame}[fragile]{Views using Types}
    \begin{minted}[fontsize=\footnotesize]{python}
def register_puppy(puppy: PuppyType) -> PuppyType:
    """
    Register a new puppy !
    """
    Puppy.objects.create(puppy)
    return JsonResponse(puppy, status=201)
    \end{minted}
\end{frame}

% Benefits
\section{Benefits}
\begin{frame}{Docs in sync with code}
    \textbf{\color{Teal}{GET}} \texttt{/puppy/}\\
    \textit{List all puppies !}\\
    Query params: \textbf{name}\\
    Response body: \textbf{List[PuppyType]}

    \vspace{3em}

    \textbf{\color{Teal}{POST}} \texttt{/puppy/}\\
    \textit{Register a new puppy !}\\
    Request body: \textbf{PuppyType}\\
    Response body: \textbf{PuppyType}
\end{frame}

\begin{frame}[fragile]{Mock the API}
    Views defined plain input parameters and output schema so that can be completely mocked.

    \begin{minted}[fontsize=\footnotesize]{python}
def list_puppy(name: PuppyName) -> typing.List[PuppyType]:
    """
    List all puppies !
    """
    pass
    \end{minted}
\end{frame}

\begin{frame}{Generate API Schema}
    All these changes made that API to expose all the information needed to automatically generate the schema.
    
    \textbf{Types} has a direct relation to \textbf{JSON Schema}, so each one can generate his own schema.

    The whole \textbf{API can be inspected} to build the schema based on standards like \textbf{OpenAPI} (former Swagger).
\end{frame}

\begin{frame}{Generic clients}
    Based on standard schemas such as \textbf{OpenAPI} it's quite easy to create generic clients for our API.
\end{frame}


\begin{frame}[standout]
    \Huge{Open source your code !}

    \Huge{\faLinux}

    \href{https://discuss.apistar.org/}{\scriptsize{https://discuss.apistar.org/}}

    \href{https://github.com/encode/apistar/tree/version-0.5.x}{\scriptsize{https://github.com/encode/apistar/tree/version-0.5.x}}

    \href{https://github.com/perdy/apistar-crud}{\scriptsize{https://github.com/perdy/apistar-crud}}

    \href{https://github.com/encode/starlette}{\scriptsize{https://github.com/encode/starlette}}

    \href{https://github.com/perdy/starlette-api}{\scriptsize{https://github.com/perdy/starlette-api}}
\end{frame}

\end{document}
