% Fields
\section{Fields}
\subsection{ArrayField}
\begin{frame}[fragile]{ArrayField}
    A field for storing lists of data that can be nested to store multi-dimensional arrays.

    \begin{block}{Example}
        \begin{minted}[fontsize=\footnotesize]{python}
from django.db import models
from django.contrib.postgres.fields import ArrayField

class Post(models.Model):
    name = models.CharField(max_length=200)
    bar = ArrayField(models.CharField(max_length=200), blank=True)

    def __str__(self):
        return self.name

>>> Post.objects.create(name='First post', tags=['thoughts', 'django'])
>>> Post.objects.create(name='Second post', tags=['thoughts'])
>>> Post.objects.create(name='Third post', tags=['tutorial', 'django'])
        \end{minted}
    \end{block}
\end{frame}
\begin{frame}[fragile]{ArrayField: Queries}
    \begin{block}{Contains (\texttt{@}\texttt{>})}<only@1>
        \begin{minted}[fontsize=\footnotesize]{python}
>>> Post.objects.filter(tags__contains=['thoughts'])
<QuerySet [<Post: First post>, <Post: Second post>]>

>>> Post.objects.filter(tags__contains=['django'])
<QuerySet [<Post: First post>, <Post: Third post>]>

>>> Post.objects.filter(tags__contains=['django', 'thoughts'])
<QuerySet [<Post: First post>]>
        \end{minted}
    \end{block}
    \begin{block}{Contained By (\texttt{<}\texttt{@})}<only@1>
        \begin{minted}[fontsize=\footnotesize]{python}
>>> Post.objects.filter(tags__contained_by=['thoughts', 'django'])
<QuerySet [<Post: First post>, <Post: Second post>]>

>>> Post.objects.filter(tags__contained_by=['thoughts', 'django', 'tutorial'])
<QuerySet [<Post: First post>, <Post: Second post>, <Post: Third post>]>
        \end{minted}
    \end{block}
    \begin{block}{Overlap (\texttt{\&}\texttt{\&})}<only@2>
        \begin{minted}[fontsize=\footnotesize]{python}
>>> Post.objects.filter(tags__overlap=['thoughts'])
<QuerySet [<Post: First post>, <Post: Second post>]>

>>> Post.objects.filter(tags__overlap=['thoughts', 'tutorial'])
<QuerySet [<Post: First post>, <Post: Second post>, <Post: Third post>]>
        \end{minted}
    \end{block}
    \begin{block}{Len}<only@2>
        \begin{minted}[fontsize=\footnotesize]{python}
>>> Post.objects.filter(tags__len=1)
<QuerySet [<Post: Second post>]>
        \end{minted}
    \end{block}
    \begin{block}{Index Transforms}<only@3>
        \begin{minted}[fontsize=\footnotesize]{python}
>>> Post.objects.filter(tags__0='thoughts')
<QuerySet [<Post: First post>, <Post: Second post>]>

>>> Post.objects.filter(tags__1__iexact='Django')
<QuerySet [<Post: First post>]>

>>> Post.objects.filter(tags__276='javascript')
<QuerySet []>
        \end{minted}
    \end{block}
    \begin{block}{Slice Transforms}<only@3>
        \begin{minted}[fontsize=\footnotesize]{python}
>>> Post.objects.filter(tags__0_1=['thoughts'])
<QuerySet [<Post: First post>, <Post: Second post>]>

>>> Post.objects.filter(tags__0_2__contains=['thoughts'])
<QuerySet [<Post: First post>, <Post: Second post>]>
        \end{minted}
    \end{block}

    \note{
        \begin{description}
            \item[First post:] thoughts, django.
            \item[Second post:] thoughts.
            \item[Third post:] tutorial, django.
        \end{description}

        Remember here the relevance of the lookups and transform defined by Django, because these will be the most widely used, so it's a good idea to \textbf{index} them.
    }
\end{frame}

\begin{frame}{ArrayField: Forms}
    \begin{block}{SimpleArrayField}
        A simple field which maps to an array. It is represented by an HTML \texttt{<input>}.
    \end{block}
    \begin{block}{SplitArrayField}
        This field handles arrays by reproducing the underlying field a fixed number of times.
    \end{block}
\end{frame}

\subsection{HStoreField}
\begin{frame}[fragile]
    \frametitle{HStoreField}

    A field for storing key-value pairs. The Python data type used is a dict that maps \emph{strings} into \emph{nullable strings}.

    \begin{onlyenv}<1>
        \begin{block}{Example}
            \begin{minted}[fontsize=\scriptsize]{python}
from django.contrib.postgres.fields import HStoreField
from django.db import models

class Dog(models.Model):
    name = models.CharField(max_length=200)
    data = HStoreField()

    def __str__(self):
        return self.name

>>> Dog.objects.create(name='Rufus',
                       data={'breed': 'labrador', 'owner': 'Bob', 'friend': 'Bobby'})
>>> Dog.objects.create(name='Meg',
                       data={'breed': 'collie', 'owner': 'Bob', 'toy': 'yellow ball'})
>>> Dog.objects.create(name='Fred', data={})
            \end{minted}
        \end{block}
    \end{onlyenv}

    \begin{onlyenv}<2>
        \begin{block}{PostgreSQL extension}
            \begin{minted}[fontsize=\scriptsize]{python}
from django.contrib.postgres.operations import HStoreExtension

class Migration(migrations.Migration):
    ...

    operations = [
        HStoreExtension(),
        ...
    ]
            \end{minted}
        \end{block}
    \end{onlyenv}

    \note {
        Remember that for using HStoreField is required to install \textbf{hstore extension}.
    }
\end{frame}
\begin{frame}[fragile]{HStoreField: Queries}
    \begin{block}{Key Lookups}<only@1>
        \begin{minted}[fontsize=\scriptsize]{python}
>>> Dog.objects.filter(data__breed='collie')
<QuerySet [<Dog: Meg>]>

>>> Dog.objects.filter(data__breed__contains='l')
<QuerySet [<Dog: Rufus>, <Dog: Meg>]>
        \end{minted}
    \end{block}
    \begin{block}{Contains (\texttt{@}\texttt{>})}<only@1>
        \begin{minted}[fontsize=\scriptsize]{python}
>>> Dog.objects.filter(data__contains={'owner': 'Bob'})
<QuerySet [<Dog: Rufus>, <Dog: Meg>]>

>>> Dog.objects.filter(data__contains={'breed': 'collie'})
<QuerySet [<Dog: Meg>]>
        \end{minted}
    \end{block}
    \begin{block}{Contained By (\texttt{<}\texttt{@})}<only@2>
        \begin{minted}[fontsize=\scriptsize]{python}
>>> Dog.objects.filter(data__contained_by={'breed': 'collie', 'owner': 'Bob'})
<QuerySet [<Dog: Meg>, <Dog: Fred>]>

>>> Dog.objects.filter(data__contained_by={'breed': 'collie'})
<QuerySet [<Dog: Fred>]>
        \end{minted}
    \end{block}
    \begin{block}{Has Key (\texttt{?})}<only@2>
        \begin{minted}[fontsize=\scriptsize]{python}
>>> Dog.objects.filter(data__has_key='toy')
<QuerySet [<Dog: Meg>]>
        \end{minted}
    \end{block}
    \begin{block}{Has Any Keys (\texttt{?}\texttt{|})}<only@2>
        \begin{minted}[fontsize=\scriptsize]{python}
>>> Dog.objects.filter(data__has_any_keys=['toy', 'friend'])
<QuerySet [<Dog: Rufus>, <Dog: Meg>]>
        \end{minted}
    \end{block}
    \begin{block}{Has Keys (\texttt{?}\texttt{\&})}<only@3>
        \begin{minted}[fontsize=\scriptsize]{python}
>>> Dog.objects.filter(data__has_keys=['breed', 'owner'])
<QuerySet [<Dog: Rufus>, <Dog: Meg>]>
        \end{minted}
    \end{block}
    \begin{block}{Keys}<only@3>
        \begin{minted}[fontsize=\scriptsize]{python}
>>> Dog.objects.filter(data__keys__overlap=['breed', 'toy'])
<QuerySet [<Dog: Rufus>, <Dog: Meg>]>
        \end{minted}
    \end{block}
    \begin{block}{Values}<only@3>
        \begin{minted}[fontsize=\scriptsize]{python}
>>> Dog.objects.filter(data__values__contains=['collie'])
<QuerySet [<Dog: Meg>]>
        \end{minted}
    \end{block}

    \note {
        \begin{description}
            \item[Rufus:] 
                \begin{description}
                    \item[breed:] labrador.
                    \item[owner:] Bob.
                    \item[friend:] Bobby.
                \end{description}
            \item[Meg:]
                \begin{description}
                    \item[breed:] collie.
                    \item[owner:] Bob.
                    \item[toy:] yellow ball.
                \end{description}
            \item[Fred:]
        \end{description}

        Keys example uses \texttt{overlap} because order is not guaranteed.
    }
\end{frame}
\begin{frame}[fragile]{HStoreField: Forms}
    \begin{block}{HStoreField}
        A field which accepts JSON encoded data for an HStoreField, casting all values (except nulls) to strings. It is represented by an HTML \texttt{<textarea>}.
    \end{block}
\end{frame}

\subsection{JSONField}
\begin{frame}[fragile]{JSONField}
    A field for storing JSON encoded data. In Python the data is represented in its Python native format: dictionaries, lists, strings, numbers, booleans and \texttt{None}.

    \begin{block}{Example}
        \begin{minted}[fontsize=\footnotesize]{python}
from django.contrib.postgres.fields import JSONField
from django.db import models

class Dog(models.Model):
    name = models.CharField(max_length=200)
    data = JSONField()

    def __str__(self):
        return self.name

>>> Dog.objects.create(name='Rufus', data={
...     'breed': 'labrador',
...     'owner': {
...         'name': 'Bob',
...         'other_pets': [{'name': 'Fishy'}],
...     },
... })
>>> Dog.objects.create(name='Meg', data={'breed': 'collie'})
        \end{minted}
    \end{block}

    \note {
        Explain the differentes between \emph{JSON} and \emph{JSONB}.
    }
\end{frame}
\begin{frame}[fragile]{JSONField: Queries}
    JSONField shares lookups with HStoreField:
    \begin{itemize}
        \item Contains (\texttt{@}\texttt{>})
        \item Contained By (\texttt{<}\texttt{@})
        \item Has Key (\texttt{?})
        \item Has Any Keys (\texttt{?}\texttt{|})
        \item Has Keys (\texttt{?}\texttt{\&})
    \end{itemize}

    \begin{block}{Key, Index and Path Lookups}
        \begin{minted}[fontsize=\scriptsize]{python}
>>> Dog.objects.filter(data__breed='collie')
<QuerySet [<Dog: Meg>]>

>>> Dog.objects.filter(data__owner__name='Bob')
<QuerySet [<QuerySet <Dog: Rufus>]>

>>> Dog.objects.filter(data__owner__other_pets__0__name='Fishy')
<QuerySet [<Dog: Rufus>]>
        \end{minted}
    \end{block}
\end{frame}
\begin{frame}[fragile]{JSONField: Forms}
    \begin{block}{JSONField}
        A field which accepts JSON encoded data for a JSONField. It is represented by an HTML \texttt{<textarea>}.
    \end{block}
\end{frame}
