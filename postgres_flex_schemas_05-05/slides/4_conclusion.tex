% Conclusion
\section{Conclusion}
\subsection{PostgreSQL as Relational and NoSQL}
\begin{frame}[standout]
    \huge{PostgreSQL as Relational and NoSQL?}

    \huge{\faDatabase}
\end{frame}

\subsection{Use cases}
\begin{frame}[fragile]{Use case: Scraper schema}
    \begin{block}{Problem}<1->
        Wants to scrape some websites crawling over search results and store all these results. These information must be accesible querying by website, date, and fuzzy matching over item name.
    \end{block}

    \begin{block}{Schema}<2->
        \begin{minted}[fontsize=\footnotesize]{python}
from django.contrib.postgres.fields import JSONField
from django.db import models

class Item(models.Model):
    website = models.URLField()
    name = models.CharField(max_length=200)
    timestamp = models.DateTimeField(auto_now_add=True)
    data = JSONField()

>>> Item.objects.annotate(
...     similarity=TrigramSimilarity('name', 'Similar item name'),
... ).filter(
...     similarity__gt=0.5,
...     timestamp__year=2017,
...     website='http://www.example.com'
... ).order_by('-similarity')
        \end{minted}
    \end{block}
\end{frame}
\begin{frame}[fragile]{Use case: Audit schema}
    \begin{block}{Problem}<1->
        Integrate an audit or logging system into an application. This system generates an entry for each request-response done by users and associates it to they. Entries must contains request and response data and metadata.
    \end{block}

    \begin{block}{Schema}<2->
        \begin{minted}[fontsize=\footnotesize]{python}
import datetime

from django.contrib.auth.models import User
from django.contrib.postgres.fields import JSONField
from django.db import models

class Entry(models.Model):
    user = models.ForeignKey(User)
    timestamp = models.DateTimeField(auto_now_add=True)
    request = JSONField()
    response = JSONField()

>>> Entry.objects.filter(user__id=1, timestamp=datetime.datetime.today())
        \end{minted}
    \end{block}
\end{frame}
